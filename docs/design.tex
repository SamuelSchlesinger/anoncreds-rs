\documentclass{article}

\usepackage{amsmath}
\usepackage{amsthm}
\usepackage{amssymb}

\title{anoncreds}
\author{Samuel Schlesinger}
\date{\today}

\newtheorem{definition}{Definition}

\begin{document}
\maketitle

\begin{definition}

An Anonymous Credit Scheme (ACS) consists of several probabilistic polynomial
time algorithms, which compose several protocols. 

The algorithms are:

\begin{enumerate}

    \item $(x, w) \leftarrow GenerateKeyPair$

        Run by the issuer before any Credit Tokens can be issued and spent,
        this algorithm generates a keypair for the Issuer.

    \begin{itemize}
        \item w: public key
        \item x: private key
    \end{itemize}

    \item $(p, req) \leftarrow RequestIssuance$

        Run by the client to generate a request for a Credit Token, this
        algorithm generates client secrets which will be used to construct the
        Credit Token, as well as a request for a Credit Token.

    \begin{itemize}
        \item p: client secrets
        \item req: issuance request
    \end{itemize}

    \item $resp \leftarrow Mint(x, req, n)$

        Run by the issuer in response to a request for a Credit Token, this
        algorithm verifies the issuance request and issues an issuance
        response, while remaining blind to the nullifier which is associated
        with this Credit Token. This algorithm fails if the issuance request is
        incorrect.

    \begin{itemize}
        \item x: the issuer's private key
        \item req: issuance request
        \item n: the number of credits which will be issued in the Credit Token
              constructed from this response
        \item resp: a response to an issuance request
    \end{itemize}

    \item $token \leftarrow IssueCredits(p, w, req, resp)$

        Run by the client upon receiving an issuance response from the issuer,
        this algorithm verifies the issuance response and forms the new Credit
        Token from the client secrets and issuance response.

    \begin{itemize}
        \item p: client secrets
        \item w: issuer public key
        \item req: issuance request
        \item resp: issuance response
        \item token: a Credit Token
    \end{itemize}

    \item $(p, sp) \leftarrow ProveSpend(token, charge)$

        Run by the client to generate a request to spend a Credit Token, this
        algorithm generates client secrets which will be used to construct the
        Credit Token, as well as a request for a Credit Token.

    \begin{itemize}
        \item token: the Credit Token we are requesting to spend
        \item charge: the number of credits we're requesting to spend
        \item p: client secrets
        \item sp: proof of correct spend
    \end{itemize}

    \item $k \leftarrow GetNullifier(sp)$

        Run by the issuer to extract the nullifier from the proof of correct
        spend. The nullifier is the random value associated with the underlying
        Credit Token which is revealed upon spending, preventing the double
        spending a Credit Token.
     
    \begin{itemize}
        \item sp: proof of correct spend
        \item k: the nullifier
    \end{itemize}

    \item $refund \leftarrow Refund(x, sp, c)$

        Run by the issuer to respond to a request to spend a Credit Token, this
        algorithm verifies the proof of correct spend of the given number of
        credits $c$ and issues a refund for the remaining credits. This
        algorithm fails if the spend proof is incorrect.

    \begin{itemize}
        \item x: the issuer's private key
        \item sp: proof of correct spend
        \item c: the number of credits the client wishes to spend
        \item refund: a refund
    \end{itemize}

    \item $token \leftarrow RefundCredits(p, sp, refund, w)$

        Run by the client upon receiving a refund from the issuer, this algorithm
        verifies the refund and forms the new Credit Token from the client secrets
        and the refund.

    \begin{itemize}
        \item p: client secrets
        \item sp: proof of correct spend
        \item refund: a refund
        \item w: issuer's public key
        \item token: a Credit Token
    \end{itemize}

\end{enumerate}

The protocols are run between the client and the server, initiated by the client:

\begin{enumerate}
    \item $Issue$

        The issuer knows their private key $x$, their public key $w$, and the number of
        credits they wish to issue, $n$.

        The client knows the public key of the issuer, $w$.

        \begin{enumerate}
            \item[] $(p, req) \leftarrow RequestIssuance$ // run by client
            \item[] // client sends req to issuer
            \item[] $resp \leftarrow Mint(x, req, n)$ // run by issuer
            \item[] // issuer sends resp to client
            \item[] $token \leftarrow IssueCredits(p, w, req, resp)$ // run by client
            \item[] return $token$
        \end{enumerate}

    \item $Spend$

        The issuer knows their private key $x$, their public key $w$, and the number of
        credits $c$ the client wishes to spend. They maintain a database $db$ which
        contains all previously seen nullifiers $k$.

        The client knows their $token$, the issuer's public key $w$, and the
        number of credits $c$ they wish to spend.

        \begin{enumerate}
            \item[] $(p, sp) \leftarrow ProveSpend(token, charge)$ // run by the client
            \item[] // client sends sp, charge to the issuer
            \item[] $refund \leftarrow Refund(x, sp, c)$ // run by issuer
            \item[] $k \leftarrow GetNullifier(sp$
            \item[] if $db.lookup(k)$, fail
            \item[] $db.insert(k)$
            \item[] // issuer sends refund to client
            \item[] $token \leftarrow RefundCredits(p, w, req, resp)$ // run by client
            \item[] return $token$
        \end{enumerate}
\end{enumerate}

\end{definition}

\begin{definition}

An Anonymous Credit System is \textbf{correct} if, for any efficient adversary
    $A$, the probability that $flag$ is set to $1$ in the following experiment
    is negligible.

\begin{enumerate}
    \item Generate keys $(w, x)$ and provide $w$ to $A$.
    \item $A$ can interact with the following oracles, with $i$ being initialized
          as $0$ and $db$ being initialized emptily:
        \begin{itemize}

            \item $Issue(n)$ runs the $Issue$ protocol between an honest issuer
                and client with input $n$ and lets $token$ be the resulting
                credits token. If $Issue$ fails, set $flag = 1$. Otherwise,
                $token_i = token$ and increment $i$.

            \item $Spend(j, c)$, which can be called only once for each $j$ and
                only when the balance $n$ of $token_j$ satisfies $c \leq n$.
                Run the $Spend$ protocol using $db$, $c$, and $token_j$,
                returning $token$.  Send $sp$, $c$, $k$ to $A$, and set $flag =
                1$ if the protocol fails. Set $token_i = token$ and increment
                $i$.

        \end{itemize}
\end{enumerate}
\end{definition}


\begin{definition}{Fiscal Soundness}

    (High level) an issuer should be assured that the total number of credits spent is less than or equal to the total number of credits issued.
\end{definition}

\begin{definition}{Anonymity}

    (High level) clients should be confident that their spends cannot be correlated to their issuances or previous spends with probability better than guessing. In particular, assuming there are multiple unspent credit tokens with greater than or equal to $c$ credits, an issuer should not have any advantage over random chance in guessing which of these tokens were spent.
\end{definition}

\end{document}
